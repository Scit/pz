\ESKDdocName{Вопросы физиологии и напряженности труда пользователей ПЭВМ}
\renewcommand{\ESKDtheColumnXIfII}{\ESKDfontII\text{Горешник~И.Д.}}
\updateStamp
\ESKDthisStyle{formII}
\section{Вопросы физиологии и напряженности труда пользователей ПЭВМ}
\subsection*{Введение}
Физиология труда~--- раздел физиологии, изучающий закономерности протекания физиологических процессов и особенности их регуляции при трудовой деятельности человека, т. е. трудовой процесс в его физиологических проявлениях. 

Физиология труда решает две основные задачи:
\begin{enumerate}
\item определяет оптимальные характеристики рабочего процесса, позволяющие достигнуть высокой производительности и эффективности труда;
\item разрабатывает мероприятия, предохраняющие человека от неблагоприятного влияния негативных факторов.
\end{enumerate}

Исходя из этих задач, физиология труда обосновывает режимы труда и отдыха в зависимости от интенсивности, экстенсивности, сложности и значимости трудовой деятельности; выясняет оптимальные и предельные возможности человека по приему, переработке и выдаче информации (например, наилучшие способы подачи зрительной, слуховой и др. информации на табло и щитах управления); определяет наиболее экономичные и наименее утомляющие виды рабочих движений.

Физиология труда определяет, оценивает и прогнозирует функциональное состояние организма человека до, во время и после трудовой деятельности; обосновывает мероприятия по рационализации труда, ведущие к повышению работоспособности человека и сохранению его здоровья.

Данная разработка вопросов физиологии и напряженности труда пользователей ПЭВМ осуществлялась в соответствии с методическими рекомендациями, на основе санитарно-эпидемиологических правил и нормативов:
\begin{itemize}
\item P 2.2.2006-05 "Руководство по оценке гигиенических факторов рабочей среды и трудового процесса. Критерии и классификации условий труда";
\item СаНПиН 2.2.2/2.4.1340-03 “Гигиенические требования к персональным электронно-вычислительным машинам и организации работы";
\item СниП 23-05-95 “Естественное и искусственное освещение”;
\item ГОСТ 12.1.003-83 "ССБТ. Шум. Общие требования безопасности";
\item ГОСТ 12.2.032-78 “ССБТ. Рабочее место при выполнении работ сидя. Общие эргономические требования”.
\end{itemize}

В ходе работы были изучены аспекты физиологии труда, актуальные для профессиональной деятельности пользователей ПЭВМ, в том числе, характерные для них условия труда. Кроме этого, был проведен анализ профессиональных заболеваний и определены мероприятия по профилактике и исключению вредных факторов, характерные для типов работ, связанных с ПЭВМ.

\subsection{Ключевые понятия физиологии труда для работ, связанных с использованием ПЭВМ}
В связи с появлением систем автоматического управления и увеличением сферы операторского умственного труда перед физиологией труда возникли новые задачи. Основные из них~--- это проблемы умственного утомления, сенсорного голода и сенсорного пресыщения. Данные проблемы возникают в том случае, когда человек работает в условиях, соответственно, интенсивного, недостаточного, или очень большого числа раздражителей, действующих на органы чувств, что ведет к уменьшению или чрезмерному повышению общего тонуса центральной нервной системы. Кроме того, требуют разрешения ставшие актуальными проблемы недостаточной двигательной активности (гиподинамия) и малых мышечных напряжений (гипокинезия), а также проблема резких нервно-эмоциональных напряжений.

С физиологической точки зрения труд есть затрата физической и умственной энергии человека, но он необходим и полезен человеку. И только во вредных условиях или при чрезмерном напряжении сил человека в той или иной форме могут проявляться негативные последствия труда. Труд принято характеризовать тяжестью и напряженностью. Для работ, связанных с ПЭВМ ключевым фактором является вторая характеристика труда, т.е. его напряженность.
Напряженность труда~--- это характеристика трудового процесса, отражающая нагрузку преимущественно на центральную нервную систему, органы чувств, эмоциональную сферу работника. К факторам, характеризующим напряженность труда, относятся: интеллектуальные, сенсорные, эмоциональные нагрузки, степень монотонности нагрузок, режим работы.

Напряженность также можно оценивать по изменению уровня функционирования соответствующих систем организма сравнительно с исходным состоянием оперативного покоя оператора.

Критериями степени напряженности является выраженное нарушение адекватности физиологических реакций, резкое снижение точности, быстродействия и надежности оператора, ведущее к дезорганизации его деятельности.

Работа с проектируемой в рамках дипломного пректа распределенной агентно-ориентированной системы доступа к базам данных напрямую связана с  ПЭВМ: как для конечных пользователей клиентской части программы, так и для системных администраторов, обслуживающих функционирование удаленных серверных узлов, вследствие чего, тема напряженности труда при работе с ПЭВМ очень актуальна и важна для данного проекта.

\subsubsection{Характер труда пользователей ПЭВМ}
Для оценки напряженности умственного труда, к которой и относится работа пользователей ПЭВМ, с физиологической точки зрения не выработаны достаточно объективные критерии. Ее можно характеризовать объемом информации, подлежащей запоминанию и (или) анализу, а также скоростью поступления информации и принятия решений, мерой ответственности за возможные ошибки при принятии решений и др.

Понятие умственного труда имеет, прежде всего, психофизиологическую основу, поскольку оно связано с деятельностью человеческого мозга.

Характер работы, связанной с использованием ПЭВМ, как правило, заключается в переработке и анализе большого объема разнообразной информации, и как следствие этого~--- мобилизация памяти и внимания, а мышечные нагрузки, как правило, незначительны. Этот труд характеризуется значительным снижением двигательной активности (гипокинезией), что может приводить к сердечно-сосудистой патологии; длительная умственная нагрузка угнетает психику, ухудшает функции внимания, памяти.

Из этого следует, что работа, связанная с использованием ПЭВМ относится к категории работ с опасными и вредными условиями труда. И здесь существует немало факторов, оказывающих негативное влияние на психофизиологическое состояния пользователя и степень напряженности его труда.

\subsubsection{Основные опасные и вредные факторы}
Список основных опасных и вредных факторов можно разбить на категории, соответствующие тем системам организма человека, на которые они оказывают свое воздействие.

Зрение. К основным факторам, отрицательно влияющим на зрение при работе с монитором, можно отнести следующие:
\begin{itemize}
\item неправильно настроенная четкость и резкость изображения;
\item неправильно настроенная яркость;
\item мерцание изображения;
\item наличие бликов.
\end{itemize}

Органы дыхания. Пользователи ПК часто сталкиваются с заболеваниями органов дыхания. Во время долгой работы компьютера корпус монитора и платы в системном блоке нагреваются и могут выделять в воздух вредные вещества. Помимо этого, компьютер создает вокруг себя электростатическое поле, притягивающее пыль, которая оседает в легких. В то же время работающий компьютер уменьшает влажность воздуха. Каждый из этих факторов пагубно влияет как на легкие, так и на весь организм в целом.

Опорно-двигательная система. У людей, проводящих много времени за компьютером, возникают проблемы, связанные с мышцами и суставами. Неподвижная напряженная поза оператора, в течение длительного времени находящегося у экрана монитора, приводит к усталости и возникновению болей в позвоночнике, шее, плечевых суставах. Кроме того, развивается мышечная слабость и происходит изменение формы позвоночника. Интенсивная работа с клавиатурой вызывает болевые ощущения в локтевых суставах, предплечьях, запястьях в кистях и пальцах рук.

Нервная система. Характерными заболеваниями пользователей ПК являются нервные расстройства. Например, мерцание экрана, шум вентиляторов сильно напрягают нервную систему.

Общее утомление нервной системы приводит к иллюзии физической усталости, снижению чувствительности органов чувств (не только зрения и слуха, но и всех остальных), нарушению координации движений и чувства равновесия, а также к нарушениям давления и спазмам сосудов.

Еще одним фактором, влияющим на нервную систему пользователя ПК, является большой поток информации, который он должен воспринимать.

Все эти факторы ведут к повышенной напряженности труда, возниконовению профессиональных заболеваний, снижению иммунитета, общего самочувствия и жизненного тонуса человека. Одним из ключевых моментов, способствующих минимизации данных факторов является правильная организация рабочего процесса.

\subsection{Организация рабочего процесса}
Организация рабочего процесса занимает центральное место, среди проблем, рассматриваемых физиология труда.
\subsubsection{Организация рабочего места}
Организация рабочего места является одной из главных задач организации рабочего процесса пользователя ПЭВМ.

Работа с компьютером характеризуется высокой напряженностью зрительной работы и достаточно большой нагрузкой на мышцы рук, спины, шеи. Поэтому большое значение имеет рациональная конструкция и расположение элементов рабочего стола. Неправильная организация рабочего места и порядка работы может приводить к заболеванием нервной системы, заболеванием костно-мышечной системы, заболевание глаз.

В соответствии с СаНПиН 2.2.2/2.4.1340-03 при организации рабочего места пользователя ПК следуем рекомендациям эргономики. Одна из важных задач эргономики~--- снизить нагрузки на организм оператора. Эргономическими аспектами проектирования видеотерминальных рабочих мест являются следующие правила:
\begin{itemize}
\item высота рабочей поверхности~--- 680~-- 800~мм.;
\item размеры пространства для ног: высота~--- не менее 600~мм., ширина~--- не менее 500~мм., глубина~--- не менее 450~-- 600~мм.;
\item расстояние от глаз пользователя до экрана~--- 600~--700~мм.;
\item характеристика рабочего кресла: высота сиденья над уровнем пола 420~-- 550~мм. ; ширина сиденья не менее 400мм., глубина~-- 400~-- 500~мм.
\item регулируемость элементов рабочего места: рабочий стул должен быть подъемно-поворотным и регулируемым по высоте и углам наклона спинки; регулирование экрана по высоте, наклону относительно вертикали;
\item клавиатура должна располагаться на поверхности стола на расстоянии 100~-- 300~мм. от края, обращенного к пользователю, или на специальной регулируемой высоте рабочей поверхности, отделенной от основной столешницы.
\item площадь на одного пользователя должна составлять не менее 6~м.$^2$, а объем~--- не менее 20~м.$^3$.
\end{itemize}

Общие рекомендации для пользователей ПЭВМ:
\begin{itemize}
\item нижней уровень экрана должен находиться на 20~см. ниже уровня глаз;
\item уровень верхней кромки экрана должен быть на высоте лба, 
спинка кресла должна поддерживать спинку пользователя;
\item угол между бедрами и позвоночником должен составлять 90~градусов.
\end{itemize}

Комфортное состояние жизненного пространства по показателям микроклимата и освещения достигается соблюдением нормативных требований. В качестве критериев комфортности ГОСТ устанавливает значения температуры воздуха в помещении, его влажности и подвижности (таблица \ref{bzhd:clima}).

\begin{center}
\begin{longtable}{|p{2.5cm}|p{3cm}|p{3.8cm}|p{3cm}|p{3.2cm}|}
\caption{Оптимальные нормы микроклимата для помещений с ПЭВМ}
\label{bzhd:clima}\\
\hline
\textbf{Период года} & \textbf{Категория работ} & \textbf{Температура воздуха, С} & \textbf{Относит. влажность воздуха, \%} & \textbf{Скорость движения воздуха, м/с} \\
\hline
\endfirsthead
\caption*{Продолжение таблицы \thetable}\\
\hline
\textbf{Период года} & \textbf{Категория работ} & \textbf{Температура воздуха, С} & \textbf{Относит. влажность воздуха, \%} & \textbf{Скорость движения воздуха, м/с} \\
\hline
\endhead
\endfoot
\hline
\endlastfoot
Холодный & Легкая - 1 & 22-24 & 40-60 & 0,1 \\ \hline
Холодный & Легкая - 2 & 21-23 & 40-60 & 0,1 \\ \hline
Теплый & Легкая - 1 & 23-25 & 40-60 & 0,1 \\ \hline
Теплый & Легкая - 2 & 22-24 & 40-60 & 0,2 \\ \hline
\end{longtable}
\end{center}

Уровень шума на рабочем месте операторов ЭВМ не должен превышать 50~дБ. Однако указанный уровень должен быть на 5~дБ ниже при выполнении напряженной работы в течение более 8 часов.

Согласно строительным нормам и правилам 23-05-95 помещение с ПЭВМ должны иметь естественное и искусственное освещение. Естественное освещение должно осуществляться через светопроемы, ориентированные преимущественно на север и северо-восток и обеспечивать коэфициенты естественной освещенности (КЕО) не ниже 1.2\% в зонах с устойчивым снежным покровом и 1.5\% на остальной территории. Искусственное освещение в помещениях эксплуатации ПЭВМ должно осуществляться системой общего равномерного освещения. Освещенность на поверхности стола в зоне размещения рабочего документа должна быть 300~-- 500~лк. Допускается установка светильников местного освещения для подсветки документов. Следует ограничивать прямую блесткость от источников освещения, при этом яркость светящихся поверхностей, находящихся в поле зрения, должна быть не более 200~кд/м$^2$.

\subsubsection{Организация режима труда}
Соблюдение правильного режима труда и отдыха играет важную роль при работе с  ПЭВМ.

Для того, чтобы в течение дня поддерживать  правильную осанку, мышцам, которые за это отвечают, необходим отдых. Это~--- мышцы спины, шеи и живота. Правильная осанка предусматривает изменение позы примерно дважды в час. Длительное пребывание в одной позе заставляет мышцы работать непрерывно без отдыха. Из-за отсутствия достаточного отдыха в мышцах накапливаются продукты распада, вызывающие болезненные ощущения. Неправильная организация режима труда приводит к повышенной напряженности трудовой деятельности и оказывает негативное влияние на здоровье человека.

В соответствии с санитарными правилами и нормами  все виды трудовой деятельности, связанные с использование компьютера, делятся на три группы:
\begin{itemize}
\item группа А: работа по считыванию информации с экрана ВДТ или ПЭВМ с предварительным запросом;
\item группа Б: работа по вводу информации;
\item группа В: творческая работа в режиме диалога с ЭВМ.
\end{itemize}

В таблице \ref{bzhd:schedule} представлены сведения о регламентированных перерывах, которые необходимо делать при работе на компьютере, в зависимости от продолжительности рабочей смены, видов и категорий трудовой деятельности.
\begin{center}
\begin{longtable}{|p{3cm}|p{2.5cm}|p{2.5cm}|p{2.4cm}|p{2.5cm}|p{2.5cm}|}
\caption{Время регламентированных перерывов при работе на компьютере}
\label{bzhd:schedule}\\
\hline
\multirow{2}{3cm}{\textbf{Категория работы с ВДТ или ПЭВМ}} & \multicolumn{3}{p{7.4cm}|}{\textbf{Уровень нагрузки за рабочую смену при видах работы с ВДТ}} & \multicolumn{2}{p{5cm}|}{\textbf{Суммарное время регламентированных перерывов, мин.}} \\
\cline{2-6}
& \textbf{Группа А, количество знаков} & \textbf{Группа Б, количество знаков} & \textbf{Группа В,
часов} & \textbf{При 8-часовой смене} & \textbf{При 12-часовой смене} \\
\hline
\endfirsthead
\caption*{Продолжение таблицы \thetable}\\
\hline
\multirow{2}{3cm}{\textbf{Категория работы с ВДТ или ПЭВМ}} & \multicolumn{3}{p{7.4cm}|}{\textbf{Уровень нагрузки за рабочую смену при видах работы с ВДТ}} & \multicolumn{2}{p{5cm}|}{\textbf{Суммарное время регламентированных перерывов, мин.}} \\
\cline{2-6}
& \textbf{Группа А, количество знаков} & \textbf{Группа Б, количество знаков} & \textbf{Группа В,
часов} & \textbf{При 8-часовой смене} & \textbf{При 12-часовой смене} \\
\hline
\endhead
\endfoot
\hline
\endlastfoot
I & до 20000 & до 15000 & до 2.0 & 30 & 70 \\ \hline
II & до 40000 & до 30000 & до 4.0 & 50 & 90 \\ \hline
III & до 60000 & до 40000 & до 6.0 & 70 & 120 \\ \hline
\end{longtable}
\end{center}

Время перерывов в таблице представлено при соблюдении санитарных правил и норм. При несоответствии фактических условий труда требованиям норм время регламентированных перерывов следует увеличить на 30\%.

\subsection{Профилактика профессиональных  растройств и заболеваний пользователей ПЭВМ}
Вторым аспектом, который рассматривает и изучает физиология труда является разработка и организация профилактических мероприятий, предохраняющих человека, в данном случае~--- пользователя ПЭВМ, от неблагоприятного влияния негативных факторов, связанных с его профессиональной деятельностью.

\subsection{Действия по профилактике заболеваний систем органов человека}
Органы дыхания. В профилактику заболеваний органов дыхания следует включить:
\begin{itemize}
\item влажную уборку и проветривание помещения;
\item установку открытых емкостей с водой, например, аквариума с рыбками.
\end{itemize}

Опорно-двигательная система. Для профилактики заболеваний опорно-двигательной системы необходимо заботиться о правильной организации рабочего места, периодически менять рабочую позу, прерывать работу выполнения комплекса физических упражнений.

Зрение. Для профилактики глазных заболеваний необходимо периодически выполнять гимнастику для глаз, массировать глазные яблоки, переводить взгляд с близко расположенных объектов на объекты за окном. Следить за чистотой поверхности экрана.

Нервная система. Профилактика нервных расстройств включает:
\begin{itemize}
\item структурирование информации для того, чтобы ее было легко найти;
\item создание резервных копий;
\item периодическая проверка компьютера на наличие вирусов;
\item качественный доступ к сети интернет,  для предотвращения нервных напряжений во время длительных ожиданий запрашиваемого контента;
\item перерывы в работе с компьютером.
\end{itemize}

\subsubsection{Профилактические методы по поддержанию оптимальных условий труда}
Данная категория включает в себя:
\begin{itemize}
\item организацию смены рода задач и нагрузок;
\item соблюдение  перерывов в работе: 5 минут через 1 час работы на дисплее или 10 минут после 2-х часов работы на дисплее. В перерыве следует выполнять комплексы физических упражнений способствующих восстановлению нормальной работоспособности глаз и мышц тела, а также помогающих снять симптомы синдрома компьютерного стресса.
\end{itemize}

\subsubsection{Профилактические мероприятия по организации рабочего места}
Для поддержания нормальной температуры и относительной влажности выполнять регулярные проветривания помещения.

На рабочих местах желательна устанавка ионизаторов воздуха, вырабатывающих заряженные ионы, которые благоприятно воздействуют на состояние человека (допустимые показатели ионизации воздуха представлены в таблице \ref{bzhd:ion}):
\begin{itemize}
\item улучшается психологическое и физическое состояние;
\item увеличивается сопротивляемость организма заболеваниям;
\item снижается количество бактерий в помещении;
\item очищается воздух от взвешенных микрочастиц;
\item ослабляется эффект, вызванный статическим электричеством.
\end{itemize}

\begin{center}
\begin{longtable}{|p{6cm}|p{5cm}|p{5cm}|}
\caption{Уровни ионизации воздуха помещений при работе с ПЭВМ}
\label{bzhd:ion}\\
\hline
\multirow{2}{*}{\textbf{Уровни ионизации}} & \multicolumn{2}{c|}{\textbf{Результаты измерений}} \\
\cline{2-3}
& \textbf{n+} & \textbf{n-} \\
\hline
\endfirsthead
\caption*{Продолжение таблицы \thetable}\\
\hline
\multirow{2}{*}{\textbf{Уровни ионизации}} & \multicolumn{2}{c|}{\textbf{Результаты измерений}} \\
\cline{2-3}
& \textbf{n+} & \textbf{n-} \\
\hline
\endhead
\endfoot
\hline
\endlastfoot
Минимально необходимое & 400 & 600 \\ \hline
Оптимальное & 1500~--3000 & 3000~--50000 \\ \hline
Максимально допустимое & 50000 & 50000 \\ \hline
\end{longtable}
\end{center}

\subsubsection{Профилактика негативного воздействия шума}
Выполнять периодическую смазку подшипников вентиляторов в системном блоке. По возможности проводить плановые замены старого оборудования, имеющего повышенный уроветь шума: жесткие диски, матричные принтеры, процессорные вентиляторы. Серверные машины и станции переносить в отдельные помещения, создавать в них дополнительную шумоизляцию.

Производить снижение уровня шума, проникающего в рабочее помещение извне, за счет увеличения звукоизоляции ограждающих конструкций, установкой пластиковых окон и дверей.

\subsection*{Заключение}
Любой прогресс в науке или технике, наряду с ярко выраженными безусловно положительными явлениями, неизбежно влечет за собой и отрицательные стороны. Вопросы компьютеризации общества сейчас стоят в ряду множества факторов, влияющих на здоровье людей.

Исследования  показали, что не столько сама компьютерная техника является непосредственным фактором негативного воздействия на организм человека, сколько неправильное ее расположение, несоблюдение элементарных гигиенических норм, касающихся труда и отдыха.

Вопросы физиологии и напряженности труда играют одну из главных ролей в организации рабочего процесса, в частности при работе пользователей и системых администраторов с разработанной в рамках данного дипломного проекта распределенной системой доступа к базам данных, помогая предохранить человека от неблагоприятных факторов, снизить нагрузку на центральную нервную систему, органы чувств, эмоцианальную сферу, позволяя полноценно использовать достижения современных информационных технологий без ущерба для здоровья и уровня жизнедеятельности.
