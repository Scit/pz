\ESKDdocName{Расчет затрат на разработку РАОС доступупа к базам данных}
\ESKDchecker{Володина~О.А.}
\updateStamp
\ESKDthisStyle{formII}
\section{Расчет затрат на разработку распределенной агентно-ориентированной системы доступа к базам данных}
\subsection*{Введение}
Целью данного дипломного проекта является разработка распределенной системы доступа к базам данных с использованием концепции программных агентов. Полученная разработка позволит иметь унифицированный доступ к инфомации, расположенной на множестве удаленных узлов.

В современном мире непрерывно происходят процессы глобализации и информационной интеграции. Они затронули и нашу страну, которая в силу географического положения и размеров вынуждена исользовать  распределенные информационные системы. Данные системы обеспечивают работу с данными, расположенными на разных серверах, различных аппаратно-программных платформах и хранящимися в различных форматах. Они легко расширяются, основаны на открытых стандартах и протоколах, обеспечивают интеграцию своих ресурсов с другими информационными системами, предоставляют пользователям простые интерфейсы. Это направление информационных технологий может успешно развиваться лишь при выполнении двух главных условий~--- адекватном развитии глобальной сетевой инфраструктуры и применении реальных технологий создания распределенных информационных систем. Второй аспект и является, по-преимуществу, объектом нашего исследования и рассмотрения.

Из этого следует, что актуальность затрагиваемой темы находится на высоком уровне. Действительно, наша страна прошла начальный этап глобальной компьютеризации. Многие решения по автоматизации малых и средних задач и проектов уже решаются адекватными средствами на достаточно высоком технологическом уровне. Но вот задачи совершенно иного качества – задачи создания глобальных распределенных информационных систем – нуждаются в осмыслении и анализе. Сложность настоящего этапа во многом предопределена традиционализмом и инерционностью мышления, выражающейся в попытке переноса средств и методов, применяемых для решения задач, обрабатываемых в рамках одного вычислительного узла, в мир глобальных систем, который живет по своим законам, требующим иных технологий.

Поведение и мотивация простых разработчиков вполне понятны и оправданы. Ставится задача~--- построить информационную систему <<клиент-сервер>> на базе локальной сети с централизованной базой данных. Для этого выбирается одна из популярных многопользовательских СУБД и какие-либо средства для быстрой разработки приложений (RAD). Далее создается сама система, которая являет собой комбинацию базы данных и обращающихся к ней приложений, в которых и реализована вся прикладная логика. И до тех пор, пока такая модель работает в ограниченном масштабе~--- все идет великолепно. Предположим, что организация, для которой выполнялась разработка, настолько выросла, что вновь возникшие задачи потребовали децентрализации хранения и обработки данных и, соответственно, развития информационной системы. Здесь и совершается ошибка. Подходы, хорошо зарекомендовавшие себя во вполне определенных условиях, автоматически переносятся в совершенно иную среду, с иными правилами жизнедеятельности. В результате система становится неработоспособной и должна быть создана заново, но уже с применением адекватных средств.

Децентрализованная система априори лишена подобных недостатков. Она отлично работает как в малых, так и средних и крупных масштабах ее использования: в первом случае отсутствуют техническая и экономическая избыточность, а во втором~--- система не приходит в непригодность и неработоспособность даже при глобальных изменениях архитектуры информационной топологии внутри организации.

Для реализации такой системы в данном проекте использовалась программная среда разработки мультиагентных систем и приложений JADE. В качестве инструмента разработки выступает объектно-ориентированный язык  программирования Java,~--- обладающий полной платформной независимостью, большим набором библиотечных функций и классов, обширной документацией и сообществом разработчиков, что несомненно оказывает положительное влияение на скорость и качество разработки, что в итоге оказывает влияние на конечную стоимость программного продукта. Разработка и функционирование разрабатываемой системы осуществляется на свободной операционной системе Linux.

\subsection{Расчет трудоемкости разработки}
Базовый показатель для определения составляющих затрат труда вычисляется по формуле:

\begin{center}
$ Q = q \cdot c (1 + p) $, где:
\end{center}

$q$~--- число операторов (исходных команд) в программном продукте;

$c$~--- коэффициент сложности программы; 

$p$~--- коэффициент коррекции программы в ходе ее разработки, зависит от точности и корректности поставленной задачи~--- принимаем равным 0,1.

Коэффициент сложности программы определяется из таблицы 1 на пересечении <<группы сложности>> и <<степени новизны>>. При этом новизна определяется по принципу:
\begin{itemize}
\item А~--- разработка  принципиально новых задач;
\item Б~--- разработка  оригинальных  программ;
\item B~--- разработка  программ  с использованием типовых решений;
\item Г~--- разовая  типовая  задача.
\end{itemize}

А сложность определяется исходя из типа решаемых задач:
\begin{itemize}
\item 1~--- алгоритмы  оптимизации и моделирования  систем;
\item 2~--- задачи  учета,  отчетности  и  статистики;
\item 3~--- стандартные  алгоритмы.
\end{itemize}

Кроме того, в таблице указан коэффициент недостаточности описания программы, который потребуется в дальнейшем.

\begin{center}
\begin{longtable}{|p{3cm}|p{3cm}|p{1.5cm}|p{1.5cm}|p{1.5cm}|p{1.5cm}|p{3cm}|}
\caption{Коэффициенты расчета трудоемкости}
\label{econom:trud}\\
\hline
\multirow{2}[4]{3cm}{\textbf{Язык }} &
\multirow{2}{3cm}{\textbf{Группа сложности}} &
\multicolumn{4}{c|}{\textbf{Степень новизны}} &
\multirow{2}{3cm}{\textbf{Коэффициент B}} \\
\cline{3-6}
 & & \textbf{А} & \textbf{Б} & \textbf{В} & \textbf{Г} &  \\
\hline
\endfirsthead
\caption*{Продолжение таблицы \thetable}\\
\hline
\multirow{2}[4]{3cm}{\textbf{Язык }} &
\multirow{2}{3cm}{\textbf{Группа сложности}} &
\multicolumn{4}{c|}{\textbf{Степень новизны}} &
\multirow{2}{3cm}{\textbf{Коэффициент B}} \\
\cline{3-6}
 & & \textbf{А} & \textbf{Б} & \textbf{В} & \textbf{Г} &  \\
\hline
\endhead
\endfoot
\hline
\endlastfoot
\multirow{3}{3cm}{Высокого уровня}
  & 1 & 1,38 & 1,26 & 1,15 & 0,69 & 1,2 \\
\cline{2-7}
  & 2 & 1,30 & 1,19 & 1,08 & 0,65 & 1,35 \\
\cline{2-7}
  & 3 & 1,20 & 1,10 & 1,00 & 0,60 & 1,5 \\ \hline

\multirow{3}{3cm}{Низкого уровня} 
  & 1 & 1,58 & 1,45 & 1,32 & 0,79 & 1,2 \\
\cline{2-7}
  & 2 & 1,49 & 1,37 & 1,24 & 0,74 & 1,35 \\
\cline{2-7}
  & 3 & 1,38 & 1,26 & 1,15 & 0,69 & 1,5 \\ \hline
\end{longtable}
\end{center}

Реализованная в дипломном проекте распределенная агентная система написана на языке высокого уровня Java, степень ее новизны обозначим коэффициентом <<Б>>, а группу сложности примем за единицу.

Таким образом, коэффициент сложности равен 1,26.

Общее число операторов в программе составляет 1287.

Базовый показатель $Q = 1,26 \cdot 1287 (1 + 0,1) = 1783,8$.

Далее, рассчитаем составляющие затраты труда, среди которых выделяют:

\begin{itemize}
\item затраты труда на подготовку и описание алгоритмов;
\item затраты труда на исследование алгоритмов;
\item затраты труда на разработку алгоритмов;
\item затраты труда на программирование и отладку.
\end{itemize}

Затраты труда на подготовку и описание задачи может определяться эмпирически или по формуле:

\begin{center}
$ t_\text{оп} = \frac{T_{min} + 4 \cdot T_\text{нв} + T_max}{6} = \frac{30 + 4 \cdot 60 + 90}{6} = 60$ (чел. час.), где:
\end{center}

$T_{max}$~--- трудоемкость операции в наиболее неблагоприятных условиях (пессимистическая оценка);

$T_{min}$~--- трудоемкость операции при благоприятных условиях (оптимистическая оценка); 

$T_\text{нв}$~--- трудоемкость операции при нормальных  условиях (наиболее вероятная оценка).

Ориентировочные величины оценки трудоемкости операции подготовки описания задачи в зависимости от числа операторов q приводятся в таблице~\ref{econom:desc}.

\begin{center}
\begin{longtable}{|p{4cm}|p{4cm}|p{4cm}|p{4cm}|}
\caption{Оценка времени подготовки описания задачи}
\label{econom:desc}\\
\hline
\textbf{q} & \textbf{$T_{min}$} & \textbf{$T_\text{нв}$} & \textbf{$T_{max}$} \\
\hline
\endfirsthead
\caption*{Продолжение таблицы \thetable}\\
\hline
\textbf{q} & \textbf{$T_{min}$} & \textbf{$T_\text{нв}$} & \textbf{$T_{max}$} \\
\hline
\endhead
\endfoot
\hline
\endlastfoot
             100 & 10 & 15 & 20 \\ \hline
             500 & 20 & 35 & 50 \\ \hline
            1000 & 25 & 50 & 75 \\ \hline
            1500 & 30 & 60 & 90 \\ \hline
            2000 & 40 & 70 & 100 \\ \hline
            2500 & 50 & 80 & 110 \\ \hline
            5000 & 70 & 110 & 150 \\ \hline
           10000 & 100 & 150 & 200 \\ \hline
\end{longtable}
\end{center}

Затраты труда на исследование алгоритма решения задачи определяются формулой:
\begin{center}
$ t_\text{ис} = \frac{Q \cdot B}{85 \cdot k}$ (чел. час.), где:
\end{center}

$Q$~--- базовый коэффициент;

$B$~--- коэффициент недостаточности описания задачи, который берется из таблицы~\ref{econom:desc}, в данном случае равен 1,2;

$k$~--- коэффициент квалификации программиста, зависит от стажа работника и определяется из таблицы~\ref{econom:skill} (принимается равным 1).

\begin{center}
    \begin{longtable}{|p{7cm}|p{10cm}|}
\caption{Коэффициенты квалификации программисты}
\label{econom:skill}\\
\hline
\textbf{Опыт работы} & \textbf{Коэффициент квалификации} \\
\hline
\endfirsthead
\caption*{Продолжение таблицы \thetable}\\
\hline
\textbf{Опыт работы} & \textbf{Коэффициент квалификации} \\
\hline
\endhead
\endfoot
\hline
\endlastfoot
             До двух лет & 0.8 \\ \hline
             2-3 года & 1 \\ \hline
             3-5 лет & 1.1-1.2 \\ \hline
             5-7 лет & 1.3-1.4 \\ \hline
             более 7 лет & 1.5-1.6 \\ \hline
\end{longtable}
\end{center}

Таким образом, время на исследование алгоритмов составляет:
\begin{center}
$ t_\text{ис} = \frac{1783,8 \cdot 1,2}{85} = 31$ (чел. час.)
\end{center}

Затраты труда на разработку архитектуры и схем алгоритмов составляют:

\begin{center}
$ t_\text{ал} = \frac{Q}{25 \cdot k} = \frac{1783,8}{25} = 71.4 $
\end{center}

Затраты труда на программирование системы по составленным алгоритмам и ее отладку:

разработка: $ t_\text{пр} = \frac{Q}{25 \cdot k} = \frac{1783,8}{25} = 71,4 $ (чел. час.)

отладка: $ t_\text{отл} = \frac{1783,8}{5} = 356,8 $ (чел. час.)

Суммарные затраты труда рассчитываются как сумма составных затрат труда:
$ t_\Sigma = t_\text{ис} + t_\text{ал} + t_\text{пр} + t_\text{отл} = 60 + 31 + 71,4 + 71,4 + 356,8 = 590,6 $ (чел. час.)

\subsection{Экономические расчеты}
\subsubsection{Расчет затрат на оборудование, амортизацию и обслуживание}
Для проведения работ по написанию и тестированию программы, с учетом специфики распределенных систем, использовались 3 компьютера и 1 роутер. Одна из машин использовалась как для непосредственного запуска и функционирования системы, так и для ее разработки, ввиду чего ее технические параметры должны существенно превышать параметры остальных <<пользовательских>> конфигураций системных блоков, используемых исключительно для запуска на них и отладки системы. Следовательно, и стоимость системного блока, используемого для разработки будет отличаться в большую сторону.

\begin{center}
\begin{longtable}{|p{4cm}|p{4cm}|p{4cm}|p{4cm}|}
\caption{Список затрат на оборудование}
\label{econom:equip}\\
\hline
\textbf{Оборудование} & \textbf{Цена (руб.)} & \textbf{Количество} & \textbf{Стоимость (руб.)} \\
\hline
\endfirsthead
\caption*{Продолжение таблицы \thetable}\\
\hline
\textbf{Оборудование} & \textbf{Цена (руб.)} & \textbf{Количество} & \textbf{Стоимость (руб.)} \\
\hline
\endhead
\endfoot
\hline
\endlastfoot
Системный блок для разработки и запуска распределенной системы & 20000 & 1 & 20000 \\ \hline
Системный блок для запуска распределенной системы & 12000 & 2 & 24000 \\ \hline
Монитор & 5000 & 3 & 15000 \\ \hline
Набор периферийного оборудования & 600 & 3 & 1800 \\ \hline
Роутер & 1000 & 1 & 1000 \\ \hline
\textbf{Итого:} & & & \textbf{61800} \\ \hline
\end{longtable}
\end{center}

Стоимость оборудования на момент покупки 61800 рублей. Норма амортизации оборудования составляет 25\% от общей стоимости оборудования. Расходы на обслуживание оборудования составляют 5\% от общей стоимости.

Сумма отчислений на амортизацию оборудования в год составит:

\begin{center}
$ A_\text{год} = 61800 \cdot 0,25 = 15450 $ (руб.)
\end{center}

\begin{center}
$ A_\text{мес} = 15450 \cdot 12 = 1287,5 $ (руб.)
\end{center}

\begin{center}
$ A_\text{разр} = 1287,5 \cdot 3,7 = 4763,75 $ (руб.)
\end{center}

Расходы на обслуживание оборудования за период разработки составляют 20\% от суммы расходов на амортизацию:

\begin{center}
$ O_\text{разр} = 4763,75 \cdot 0,2 = 952,75 $ (руб.)
\end{center}

\subsubsection{Расчет затрат на электроэнергию}
Стоимость расходов на электроэнергию рассчитывается по формуле:

\begin{center}
$ \text{C}_\text{эл} = W \cdot \text{C}_\text{эн} \cdot t$, где:
\end{center}

$W$~--- потребляемая мощность оборудования (кВт);

$\text{С}_\text{эн}$~--- стоимость 1 кВт-час энергии;

$T$~--- время работы оборудования (час.).

Расчет затрат на электроэнергию определяется исходя из мощности оборудования, времени его работы и стоимости 1 кВт-часа энергии.

Стоимость 1 кВт-часа энергии при данного дипломного проекта составляла: $ \text{С}_\text{эн} = 3,5$ (руб.)

Расчет затрат на электроэнергию приводится в таблице~\ref{econom:electr}:

\begin{center}
\begin{longtable}{|p{3.6cm}|p{3cm}|p{3cm}|p{3cm}|p{3cm}|}
\caption{Список затрат на электроэнергию}
\label{econom:electr}\\
\hline
\textbf{Оборудование} & \textbf{Мощность (кВт.)} & \textbf{Время эксплуатации (час.)} & \textbf{Количество} & \textbf{Сумма (руб.)} \\
\hline
\endfirsthead
\caption*{Продолжение таблицы \thetable}\\
\hline
\textbf{Оборудование} & \textbf{Мощность (кВт.)} & \textbf{Время эксплуатации (час.)} & \textbf{Количество} & \textbf{Сумма (руб.)} \\
\hline
\endhead
\endfoot
\hline
\endlastfoot
Системный блок для разработки и запуска распределенной системы & 0,4 & 500 & 2 & 1400 \\ \hline
Системный блок для запуска распределенной системы & 0,3 & 300 & 2 & 630 \\ \hline
Монитор системного блока для запуска системы & 0,02 & 500 & 1 & 35 \\ \hline
Монитор системного блока для разработки и запуска системы & 0,02 & 300 & 2 & 42 \\ \hline
Роутер & 0,01 & 500 & 1 & 17,5 \\ \hline
\textbf{Итого:} & & & & \textbf{2124,5} \\ \hline
\end{longtable}
\end{center}

\subsubsection{Расчет расходов на оплату труда и социальные отчисления}
Заработная плата складывается из двух составляющих: основной заработной платы и дополнительной.

Основная заработная плата рассчитывается по формуле:

\begin{center}
$ \text{З}_\text{осн} = \frac{t_\Sigma}{t_\text{ср} \cdot 8} \cdot ТС$ (руб.), где:
\end{center}

$t_\Sigma$~--- суммарные затраты труда, расчитанное ранее;

$t_\text{ср}$~--- среднее число дней в месяце, равно 21 дню, умножается на количество часов в рабочем дне  –  8;

$\text{ТС}$~--- тарифная ставка.

Тарифная ставка представляет собой оплату труда установленную работодателем для инженера-программиста равную 18000 рублей ежемесячно.

Таким образом, основная заработная плата будет составлять:

\begin{center}
$ \text{З}_\text{осн} = \frac{t_\Sigma}{t_\text{ср} \cdot 8} \cdot ТС = 63278,6$ (руб.), где:
\end{center}

Дополнительная заработная плата составляет 20\% от основной заработной платы, рассчитывается по формуле:
\begin{center}
$ \text{З}_\text{доп} = \text{З}_\text{осн} \cdot 0,2 = 12655,7 $ (руб.)
\end{center}

Суммарная заработная плата (или фонд заработной платы, ФЗП) вычисляется как сумма основной и дополнительной заработных плат по формуле:

\begin{center}
$ \text{ФЗП} = \text{З}_\text{осн} + \text{З}_\text{доп} = 63278,6 + 12655,7 = 75934,3$ (руб.)
\end{center}

Отчисления на социальное страхование составляют 30\% от всей заработной платы (24.07.2009 № 212-ФЗ), вычисляются по формуле:

\begin{center}
$ CC = \text{ФЗП} \cdot 0,3 = 22780,3$ (руб.)
\end{center}

\subsubsection{Прочие расходы}
Сюда включены расходы, не попадающие в другие категории.

Для установки необходимого программного обеспечения, документации, а так же тестирования разрабатываемого проекта необходимо подключение к сети интернет. Ежемесячная оплата составляет 300 рублей. Оплата за весь период разработки составит:

\begin{center}
$ \text{И}_\text{разр} = 300 \cdot 3,7 = 1110$ (руб.)
\end{center}

Помимо этого, необходим флеш-накопитель для удобного обмена данными между компьютерами и листы формата А4 для ведения заметок, записей и составления черновых схем от руки.

Свод расходов этой категории приведен в таблице~\ref{econom:misc}:

\begin{center}
\begin{longtable}{|p{9.2cm}|p{2.3cm}|p{2.3cm}|p{2.3cm}|}
\caption{Список затрат на прочие нужды}
\label{econom:misc}\\
\hline
\textbf{Позиция} & \textbf{Цена (руб.)} & \textbf{Количество} & \textbf{Стоимость} \\
\hline
\endfirsthead
\caption*{Продолжение таблицы \thetable}\\
\hline
\textbf{Позиция} & \textbf{Цена (руб.)} & \textbf{Количество} & \textbf{Стоимость} \\
\hline
\endhead
\endfoot
\hline
\endlastfoot
Абонентская плата за интернет & 300 & 3,7 & 1110 \\ \hline
Флеш-накопитель & 450 & 1 & 450 \\ \hline
Бумага & 200 & 1 & 200 \\ \hline
\textbf{Итого:} & & & \textbf{1760} \\ \hline
\end{longtable}
\end{center}

\subsubsection{Общая смета затрат}
Приведем общую смету затрат на выполнение дипломного проекта, используя данные расходов по каждой категории (таблица~\ref{econom:all}).

\begin{center}
\begin{longtable}{|p{9.4cm}|p{3.5cm}|p{3.5cm}|}
\caption{Общая смета затрат}
\label{econom:all}\\
\hline
\textbf{Категория затрат} & \textbf{Сумма (руб.)} & \textbf{Удельный вес (\%)} \\
\hline
\endfirsthead
\caption*{Продолжение таблицы \thetable}\\
\hline
\textbf{Категория затрат} & \textbf{Сумма (руб.)} & \textbf{Удельный вес (\%)} \\
\hline
\endhead
\endfoot
\hline
\endlastfoot
Расходы на оборудование & 64023,09 & 38 \\ \hline
Затраты на оплату труда & 75934,3 & 44,5 \\ \hline
Отчисления на социальные нужды & 22780,3 & 13,2 \\ \hline
Амортизационные отчисления и расходы на обслуживание & 6264,25 & 3,4 \\ \hline
Прочие расходы & 1760,00 & 0,9\\ \hline
\textbf{Итого:} & \textbf{170761,4} & \textbf{100} \\ \hline
\end{longtable}
\end{center}

\subsection*{Выводы}
Наибольший удельный вес (44,8 \%) в смете затрат занимают расходы на оплату труда, затем идут затраты на материальные нужды (37 \%). Но если учесть, что тема, исследуемая в данной работе, практически не изучена и не реализована в виде готовых приложений в нашей стране, и к тому же довольно актуальна сегодня, то сумма затрат по данным статьям не может являться слишком высокой.

Несомненным экономическим плюсом дипломного проекта является полное использование свободного программного обеспечения: опреационная система, компилятор, дополнительные программные библиотеки, сервер базы данных~--- предоставляются по CPL и подобным лицензиям, предоставляемым ПО на бесплатной и безвозмездной основы. В качестве интегрированной среды разработки использовался Intellij IDEA Comminity Edition~--- бесплатной версии расширенного пакета, достаточной для разработки подавляющего большинства проектов.

Кроме всего прочего, внедрение разработанной системы и использование ее вместо <<классической>> централизованной системы хранения, обработки и выдачи информации позволит получить следующие неоспоримые преимущества:

\begin{enumerate}
\item Открытость~--- возможность расширения системы путем добавления новых ресурсов. Тем самым снижаются издержки на поиск, реализацию и внедрение технологий по расширению системы в случае необходимости.
\item Отказоустойчивость~--- когда наличие нескольких компьютеров позволяет дублировать информацию и повышать устойчивость к некоторым аппаратным и программным ошибкам. Распределенные системы в случае ошибки могут поддерживать частичную функциональность. Полный сбой в работе системы происходит только при сетевых ошибках.
Данный аспект имеет как социальную, так и экономическую значимость. Простои во время отказа централизованного узла зачастую приводят к простоям на всем производстве, снижая тем самым экономические показатели предприятия. Кроме этого, сбои в работе централизованных систем повышают напряженность труда у сотрудников, выбивая их из привычного ритма труда, заставляя лишний раз беспокоиться о сохранности и целостности хранимых данных.
\item Прозрачность~--- когда пользователям предоставляется полный доступ к ресурсам в системе, в то же время от них скрыта информация о распределении ресурсов по системе, что дает возможность абстрагироваться от взаимодействия с изолированными централизованными элементами системы, а сконцентрировать свое внимание на выполняемой работе, возможность получать нужные данные, не заботясь о месте их физического расположения.
\end{enumerate}

Таким образом, затраты на выполнение дипломного проекта можно считать обоснованными, т.к. в результате была изучена тема разработки распределенных систем с использованием агентно-ориентированного подхода; выполнена конкретная реализация такой системы на платформе JADE, которая может использоваться как самостоятельно, так и основой для изучения и развития данной области в дальнейнем или созданию подобных реализаций по заказу предприятий, ориентированных на выполнение конкретных задач.
