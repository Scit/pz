\ESKDdocName{Введение}
\updateStamp
\ESKDthisStyle{formII}
\section*{Введение}
Процессы мировой глобализации неразрывно связаны с развитием вычислительной техники и информационных технологий. Классические задачи, решаемые в рамках обособленного вычислительного узла, вытесняются задачами, требующими взаимной коммуникации и обмена ресурсами между инфомационными системами. В свою очередь это ведет к постоянному увеличению требований к имеющимся подходам и технологиям, предоставляющим подобные решения.

Одним из ведущих направлений в этой области является создание распределенных вычислительных систем, способных решать сложные и масштабные задачи путем взаимного сотрудничества и взаимодействия. Однако не любая совокупность объединеных компьютеров способна предоставить эффективный механизм решения крупных научных, производственных и технологических задач, поскольку, как показывают исследования, от подобных систем в процессе и принципах их функционирования требуется обеспечение следующего набора основных требований:

\begin{enumerate}
\item не требовать использования дополнительного дорогостоящего коммутационного оборудования;
\item быть достаточно простой в настройке и эксплуатации; не требовать для постоянного обслуживания высококлассных специалистов;
\item быть <<прозрачной>> для конечного пользователя, т.е. должна скрывать от него все тонкости ее функционирования;
\item быть <<самонастраиваемой>>, т.е. уметь поддерживать себя в активном состоянии в любых ситуациях (за исключением форсмажорных) без вмешательства человека;
\item поддерживать выполнение задач широкого класса;
\item быть надежно защищенной от вторжения извне.
\end{enumerate}

Вышеприведенный набор требований как нельзя лучше реализуется в рамках современной и перспективной концепции программных агентов, разработанных для решения сложных и масштабных проблем и задач. Одной из попыток показать это на реальном примере и является создение распределенной системы доступа к базам данных на основе агентно-ориентированного подхода в рамках данного курсового проекта. Перед этим будут описаны общие положения и теоретические основы данной предметной области, выполнено краткое описание современных подходов для решения подобных задач, их архитектурных особенностей, ключевых преимуществ и недостатков.

Также будет подробно описан агентно-ориентированный подход к программированию, показаны механизмы функционирования агентов, изучены их основные свойства и методы. Кроме того, будут рассмотрены принципы взаимодействия агентов между собой и их жизненные циклы.

Заключительным этапом проекта станет процесс непосредственной реализации работоспособной системы на основе программной среды разработки мультиагентных систем и приложений JADE, способной автономно функционировать в рамках поставленной задачи, предоставляя конечному пользователю унифицируемый доступ к имеющейся инфомации, расположенной на множестве компьютеров в сети.

Стоит отметить высокую актуальность рассматриваемой тематики, ее востребованность на современном рынке инфомационных технологий и программного обеспечения, а также недостаточную изученность и распространенность в широких кругах как в нашей стране, так за рубежом. Данный дипломный проект может послужить еще одним, пусть и небольшим вкладом среди пока еще узкого круга работ и материалов в области изучения, развития и реализации подхода построения распределенных вычислительных систем с использованием технологии программных агентов.
