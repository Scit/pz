\ESKDdocName{Реферат}
\updateStamp
\ESKDthisStyle{formII}
\section*{Реферат}
\addcontentsline{toc}{section}{Реферат}
Пояснительная записка содержит \_ листов формата А4, \_ рисунков, \_ таблиц, \_ листов формата А1, \_ плакатов формата А1, \_ источников, \_ приложений.

Распределенная агентно-ориентированная система доступа к базам данных.

РАСПРЕДЕЛЕННЫЕ ВЫЧИСЛИТЕЛЬНЫЕ СИСТЕМЫ, ПРОГРАММНЫЕ АГЕНТЫ, JADE, JAVA, БАЗЫ ДАННЫХ

Объектами исследования являются: распределенные системы, программные агенты, автоматизированный подход к сборке и развертыванию проектов.

Цель работы~--- разработка распределенной системы доступа к базам данных на основе технологии программных агентов.

В результате проведенной работы был реализован проект, удовлетворяющий поставленной задаче.

В пояснительной записке выполнено описание базовых понятий теории распределенных вычислительных систем, сделан обзор современных подходов к построению подобных решений,  представлено изложение принципов  автоматизированной сборки и развертывания приложений и рассмотрена работа с фреймворком Maven.

Графическая часть содержит модели UML.

\newpage
\section*{Список терминов и сокращений}
АОП~--- агентно-ориентированное программирование.

МАС~--- мультиагентная система.
 
ВС~--- вычислительная система.

СУБД~--- система управления базами данных.

ПЭВМ~--- персональная электронно-вычислительная машина.

UML (Unified Modeling Language)~--- унифицированный язык моделирования.
